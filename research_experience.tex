\vspace{6pt}

\begin{itemize}

\item[]{\cventry{August 2017--Present}{Space Technology Research Fellow}{NASA}{Dr. Badger}{}{\vspace{3pt}As a NSTRF recipient, I participate in a visiting technologist experience each year. I use this time to apply my research to real-world robotics problems at NASA. Specifically, I work on Robonaut 2 and astrobee to apply task-motion planning, multi-modal planning, multi-agent planning and formal synthesis. Throughout the year, we collaborate on research projects to inform research directions that will be useful to NASA and to rapidly test our approaches on cutting-edge problems.}}

\vspace{6pt}

\item[]{\cventry{August 2016--Present}{Research Assistant}{Rice University}{Dr. Kavraki}{}{\vspace{3pt}As a Ph.D. student at Rice, I conduct research in robotics, specifically in task and motion planning and in formal methods for robotics. I have conducted research on a variety of topics including: Applications of Machine Learing to TMP to improve scalability; theoretical work in Formal Methods to ensuring or verifying correctness and safety; applied work in using FM to enhance robot autonomy while ensuring safety; and research in multi-robot planning and control as well as multi-robot task and motion planning. Current work examines: improving scalability, particularly with multiple robots; using stochastic games to model the robot collaboration more accurately; combining machine learning with formal methods to provide formal guarantees on the result.}}

\vspace{6pt}

\item[]{\cventry{May 2016--August 2016}{Google Summer of Code}{LLVM/Linux}{Jan-Simon Moller}{}{\vspace{3pt}Worked on the Clang static-analyzer for the Linux Kernel. The project was hosted under the linux foundation, the specific project being LLVMLinux. I implemented checkers for the linux kernel.  (Exact contributions can be found for user andrewmw94 at \url{https://github.com/andrewmw94/llvm_clang_GSoC})}}

\vspace{6pt}

\item[]{\cventry{September 2013--May 2016}{Research Assistant}{Catholic University of America}{Dr. Plaku}{}{\vspace{3pt}Motion Planning: Conducted research in robot motion planning with an emphasis on motion planning for high-dimensional systems with nonlinear dynamics.  Discrete algorithms are combined with sampling-based motion planning algorithms to increase the performance.  The work shows that it is possible to gain significant speedups over other state of the art methods (eg. RRT and SYCLOP) by incorporating feedback from the continuous space into the discrete space.  This work is resulted in a publication.
Linear Temporal Logic Multi-Robot Planning:  Conducted research in motion planning involving multiple robots, multiple goals, and requirements for reaching the goals specified by a proposition of Linear Temporal Logic.  The continuous planner is guided by a discrete planner that must both satisfy the proposition and find short trajectories for the robots while avoiding collisions between robots.}}

\vspace{6pt}

\item[]{\cventry{May 2015--August 2015}{NSF REU}{DIMACS, Rutgers University}{Dr. Bekris}{}{\vspace{3pt}Probabilistic Near-Optimality:  Bekris and Dobson had proved finite time Probabilistic Near-Optimality of a planning algorithm similar to PRM; however, their proof did not include tree-based planners such as RRT.  I extended the proof to include such planners.  This work will be published as an addendum to their paper.
Motion Planning using Homotopic Constraints:  Conducted research in robotics motion planning for dynamic systems.  This further explores the advantages to be gained by coupling discrete algorithms with continuous planners.  The focus in this work is on finding shortest paths in different homotopic classes and using these to guide a sampling-based planner, by choosing a homotopic class and following the corresponding shortest path.}}

\vspace{6pt}

\item[]{\cventry{May 2014--August 2014}{Google Summer of Code}{MLPACK}{Dr. Curtin}{}{\vspace{3pt}Work on the MLPACK library for Machine Learning, specifically on structures and algorithms for nearest-neighbor searches.  My contributions were implementing R-trees and variants as well as associated search algorithms.  (Exact contributions can be found for user andrewmw94 at \url{https://github.com/mlpack/mlpack})}}

\vspace{6pt}

\item[]{\cventry{May 2013--August 2013}{NSF REU}{Florida International University}{Dr. Sun}{}{\vspace{3pt}Research in Software Defined Networking.  The work centered around the difficulties facing administrators of large networks and the possibility of adding a field to a packet header so that administration could be done easily and rules enforced efficiently. This work resulted in a publication.}}

\end{itemize}
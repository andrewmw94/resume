%% start of file `template.tex'.
%% Copyright 2006-2013 Xavier Danaux (xdanaux@gmail.com).
%
% This work may be distributed and/or modified under the
% conditions of the LaTeX Project Public License version 1.3c,
% available at http://www.latex-project.org/lppl/.


\documentclass[12pt,a4paper,roman]{moderncv}        % possible options include font size ('10pt', '11pt' and '12pt'), paper size ('a4paper', 'letterpaper', 'a5paper', 'legalpaper', 'executivepaper' and 'landscape') and font family ('sans' and 'roman')

% modern themes
\moderncvstyle{classic}                            % style options are 'casual' (default), 'classic', 'oldstyle' and 'banking'
\moderncvcolor{blue}                                % color options 'blue' (default), 'orange', 'green', 'red', 'purple', 'grey' and 'black'
%\renewcommand{\familydefault}{\sfdefault}         % to set the default font; use '\sfdefault' for the default sans serif font, '\rmdefault' for the default roman one, or any tex font name
%\nopagenumbers{}                                  % uncomment to suppress automatic page numbering for CVs longer than one page

% character encoding
\usepackage[utf8]{inputenc}                       % if you are not using xelatex ou lualatex, replace by the encoding you are using
%\usepackage{CJKutf8}                              % if you need to use CJK to typeset your resume in Chinese, Japanese or Korean

% adjust the page margins
\usepackage[scale=0.87]{geometry}
% \setlength{\hintscolumnwidth}{3cm}                % if you want to change the width of the column with the dates
%\setlength{\makecvtitlenamewidth}{10cm}           % for the 'classic' style, if you want to force the width allocated to your name and avoid line breaks. be careful though, the length is normally calculated to avoid any overlap with your personal info; use this at your own typographical risks...

\usepackage{import}


%\usepackage{hyperref}
%\urlstyle{same}

% personal data
\name{}{Andrew M. Wells}
% \title{R\'{e}sum\'{e}}                               % optional, remove / comment the line if not wanted
%\address{my address, line 1, line 2, line 3, postcode}{}{}% optional, remove / comment the line if not wanted; the "postcode city" and and "country" arguments can be omitted or provided empty
\phone[mobile]{+845 702 3699}                   % optional, remove / comment the line if not wanted
%\phone[fixed]{01234 123456}                    % optional, remove / comment the line if not wanted
%\phone[fax]{+3~(456)~789~012}                      % optional, remove / comment the line if not wanted
\email{andrew.wells@rice.edu}                               % optional, remove / comment the line if not wanted
\homepage{www.andrewmwells.com}                         % optional, remove / comment the line if not wanted
%\extrainfo{additional information}                 % optional, remove / comment the line if not wanted
%\photo[64pt][0.4pt]{picture}                       % optional, remove / comment the line if not wanted; '64pt' is the height the picture must be resized to, 0.4pt is the thickness of the frame around it (put it to 0pt for no frame) and 'picture' is the name of the picture file
%\quote{Some quote}                                 % optional, remove / comment the line if not wanted

% to show numerical labels in the bibliography (default is to show no labels); only useful if you make citations in your resume
%\makeatletter
%\renewcommand*{\bibliographyitemlabel}{\@biblabel{\arabic{enumiv}}}
%\makeatother
%\renewcommand*{\bibliographyitemlabel}{[\arabic{enumiv}]}% CONSIDER REPLACING THE ABOVE BY THIS

% bibliography with mutiple entries
%\usepackage{multibib}
%\newcites{book,misc}{{Books},{Others}}
%----------------------------------------------------------------------------------
%            content
%----------------------------------------------------------------------------------
\begin{document}
%\begin{CJK*}{UTF8}{gbsn}                          % to typeset your resume in Chinese using CJK
%-----       resume       ---------------------------------------------------------
\makecvtitle

% 
\small{I am a $5^{th}$ year Ph.D. Student at Rice University advised by Dr. Lydia Kavraki. I have passed my defense, and my degree will by confered in August. In 2017, I recieved a NASA Space Technology Research Fellowship (now called Space Technology Research Grant) to conduct research in formal methods and robotics. I have applied my research at NASA; collaborating with the Robotics and Intelligence for Human Spacecraft (RIHS) team at the Lyndon B. Johnson Space Center (JSC) and with researchers at NASA Ames. I have also collaborated with researchers at other labs and universities throughout the world. I have experience leading two small projects of three or four researchers while mentoring a Ph.D. student at Rice University, including choosing high-level goals, assigning tasks, and scheduling subgoals to ensure our team meets deadlines.}

% 
\small{I am a $5^{th}$ year Ph.D. Student at Rice University advised by Dr. Lydia Kavraki. I will graduate this May. In 2017, I recieved a NASA Space Technology Research Fellowship (now called Space Technology Research Grants (NSTRG)) to conduct research in Task and Motion Planning (TMP) and in Formal Methods (FM) for Robotics. I have applied my research at NASA; collaborating with the Robotics and Intelligence for Human Spacecraft (RIHS) team at the Lyndon B. Johnson Space Center (JSC) and with researchers at NASA Ames. I have collaborated with the RIHS team at JSC since 2017 as well as researchers at other labs and universities throughout the world. I have experience leading two small projects of three or four researchers while mentoring a junior Ph.D. student at Rice University, including choosing high-level goals, assigning tasks, and scheduling subgoals to ensure our team meets deadlines.}

% 
\small{I am a $5^{th}$ year Ph.D. Student at Rice University advised by Dr. Lydia Kavraki. I will graduate this academic year. In 2017, I recieved a NASA Space Technology Research Fellowship (now called Space Technology Research Grants (NSTRG)) to conduct research in Task and Motion Planning (TMP) and in Formal Methods (FM) for Robotics. This research includes numerous publications, as well as implementations on physical robots. The research spans an array of topics including (a) applications of Machine Learing to TMP to improve scalability, (b) theoretical work in FM to ensuring or verifying correctness and safety, (c) applications of FM to robotics to enhance robot autonomy while ensuring safety and (d) research in multi-robot planning and control as well as multi-robot task and motion planning. I have applied my research at NASA; collaborating with the Robotics and Intelligence for Human Spacecraft (RIHS) team at the Lyndon B. Johnson Space Center (JSC) and with researchers at NASA Ames. I have collaborated with the RIHS team at JSC since 2017 as well as researchers at other labs and universities throughout the world. I have experience leading two small projects of three or four researchers while mentoring a junior Ph.D. student at Rice University, including choosing high-level goals, assigning tasks, and scheduling subgoals to ensure our team meets deadlines.}


\small{I am currently a senior software engineer on Tesla's Autopilot team. I received my Ph.D. at Rice University advised by Lydia Kavraki and Moshe Vardi. I was a NASA Space Technology Research Fellow. My research is in task and motion planning, ML for planning and applying formal methods to robotics.}


\section{Education}

\vspace{3pt}

\cventry{2016--2021}{Ph.D. in Computer Science}{Rice University, Drs. Lydia E. Kavraki and Moshe Y. Vardi}{}{}{}
\vspace{0.7em}
\cventry{2016--2019}{M.S. in Computer Science}{Rice University, Dr. Lydia E. Kavraki}{}{}{}
\vspace{0.7em}
\cventry{2012--2016}{B.S. in Computer Science}{Catholic University of America}{}{\textit{Magna Cum Laude}}{}  % arguments 3 to 6 can be left empty
\vspace{0.7em}
\cventry{2012--2016}{Ph.B. in Philosophy}{Catholic University of America}{}{\textit{Magna Cum Laude}}{}


\section{Experience}

% \vspace{6pt}

\begin{itemize}

\item[]{\cventry{August 2017--Present}{Space Technology Research Fellow}{NASA}{Dr. Badger}{}{\vspace{3pt}As a NSTRF recipient, I participate in a visiting technologist experience each year. I use this time to apply my research to real-world robotics problems at NASA. Specifically, I work on Robonaut 2 and astrobee to apply task-motion planning, multi-modal planning, multi-agent planning and formal synthesis. Throughout the year, we collaborate on research projects to inform research directions that will be useful to NASA and to rapidly test our approaches on cutting-edge problems.}}

\vspace{6pt}

\item[]{\cventry{August 2016--Present}{Research Assistant}{Rice University}{Dr. Kavraki}{}{\vspace{3pt}As a Ph.D. student at Rice, I conduct research in robotics, specifically in task and motion planning and in formal methods for robotics. I have conducted research on a variety of topics including: Applications of Machine Learing to TMP to improve scalability; theoretical work in Formal Methods to ensuring or verifying correctness and safety; applied work in using FM to enhance robot autonomy while ensuring safety; and research in multi-robot planning and control as well as multi-robot task and motion planning. Current work examines: improving scalability, particularly with multiple robots; using stochastic games to model the robot collaboration more accurately; combining machine learning with formal methods to provide formal guarantees on the result.}}

\vspace{6pt}

\item[]{\cventry{May 2016--August 2016}{Google Summer of Code}{LLVM/Linux}{Jan-Simon Moller}{}{\vspace{3pt}Worked on the Clang static-analyzer for the Linux Kernel. The project was hosted under the linux foundation, the specific project being LLVMLinux. I implemented checkers for the linux kernel.  (Exact contributions can be found for user andrewmw94 at \url{https://github.com/andrewmw94/llvm_clang_GSoC})}}

\vspace{6pt}

\item[]{\cventry{September 2013--May 2016}{Research Assistant}{Catholic University of America}{Dr. Plaku}{}{\vspace{3pt}Motion Planning: Conducted research in robot motion planning with an emphasis on motion planning for high-dimensional systems with nonlinear dynamics.  Discrete algorithms are combined with sampling-based motion planning algorithms to increase the performance.  The work shows that it is possible to gain significant speedups over other state of the art methods (eg. RRT and SYCLOP) by incorporating feedback from the continuous space into the discrete space.  This work is resulted in a publication.
Linear Temporal Logic Multi-Robot Planning:  Conducted research in motion planning involving multiple robots, multiple goals, and requirements for reaching the goals specified by a proposition of Linear Temporal Logic.  The continuous planner is guided by a discrete planner that must both satisfy the proposition and find short trajectories for the robots while avoiding collisions between robots.}}

\vspace{6pt}

\item[]{\cventry{May 2015--August 2015}{NSF REU}{DIMACS, Rutgers University}{Dr. Bekris}{}{\vspace{3pt}Probabilistic Near-Optimality:  Bekris and Dobson had proved finite time Probabilistic Near-Optimality of a planning algorithm similar to PRM; however, their proof did not include tree-based planners such as RRT.  I extended the proof to include such planners.  This work will be published as an addendum to their paper.
Motion Planning using Homotopic Constraints:  Conducted research in robotics motion planning for dynamic systems.  This further explores the advantages to be gained by coupling discrete algorithms with continuous planners.  The focus in this work is on finding shortest paths in different homotopic classes and using these to guide a sampling-based planner, by choosing a homotopic class and following the corresponding shortest path.}}

\vspace{6pt}

\item[]{\cventry{May 2014--August 2014}{Google Summer of Code}{MLPACK}{Dr. Curtin}{}{\vspace{3pt}Work on the MLPACK library for Machine Learning, specifically on structures and algorithms for nearest-neighbor searches.  My contributions were implementing R-trees and variants as well as associated search algorithms.  (Exact contributions can be found for user andrewmw94 at \url{https://github.com/mlpack/mlpack})}}

\vspace{6pt}

\item[]{\cventry{May 2013--August 2013}{NSF REU}{Florida International University}{Dr. Sun}{}{\vspace{3pt}Research in Software Defined Networking.  The work centered around the difficulties facing administrators of large networks and the possibility of adding a field to a packet header so that administration could be done easily and rules enforced efficiently. This work resulted in a publication.}}

\end{itemize}

\vspace{5pt}

{\cventry{2021--Present}{Senior Software Engineer}{Tesla}{Ashok Elluswamy}{}{
\begin{itemize}
    \item[] Work on planning and controls team for Autopilot.
\end{itemize}
}

\vspace{5pt}

{\cventry{2017--2021}{Space Technology Research Fellow}{NASA}{Dr. Julia Badger}{}{
\begin{itemize}
    \item[] Work at NASA JSC for 10+ weeks each year. Integrate research with large (several million lines of code) C++ code base.
    \item[] Apply research in task-motion planning and formal synthesis to Robonaut 2 and Astrobee (See [Kingston ICRA 2020])
\end{itemize}
}

\vspace{5pt}

\cventry{2016--2021}{Graduate Research Assistant}{Rice University}{Dr. Lydia E. Kavraki}{}{
\begin{itemize}
\item[] Apply Machine Learning to task-motion planning to improve scalability (See [Wells RAL 2020])
\item[] Develop finite probabilistic synthesis and applying the same to robot manipulation\\
(See \url{https://github.com/andrewmw94/ltlf_prism} and [Wells EPTCS 2020])
\item[] Mentor graduate student's research in multi-robot planning and control as well as multi-robot task-motion planning (See [Pan IROS 2020])
\item[] Develop tool for stochastic games to model human-robot collaboration
\end{itemize}
}

\vspace{5pt}

\cventry{Summer 2016}{Google Summer of Code}{LLVM/Linux}{Jan-Simon M\"oller}{}{
\begin{itemize}
\item[] Implemented checkers using the Clang static-analyzer for the Linux Kernel\\
(See user andrewmw94 at \url{https://github.com/andrewmw94/llvm_clang_GSoC})
\end{itemize}
}

\vspace{5pt}

\cventry{2013--2016}{Research Assistant}{Catholic University of America}{Dr. Erion Plaku}{}{
\begin{itemize}
\item[] Use discrete leads to motion plan for high-dimensional systems with nonlinear dynamics (See [Wells TAROS 2015])
\end{itemize}
}

\vspace{5pt}

\cventry{Summer 2015}{NSF REU}{DIMACS, Rutgers University}{Dr. Kostas Bekris}{}{
\begin{itemize}
\item[] Extended a proof of probabilistic near-optimality from $PRM^*$ to tree-based planners
\item[] Use shortest paths in different homotopic classes to guide kinodynamic motion planning
\end{itemize}
}
\vspace{5pt}

\cventry{Summer 2014}{Google Summer of Code}{MLPACK}{Dr. Ryan Curtin}{}{
\begin{itemize}
\item[] Implement R-Trees and variants for nearest-neighbor searches\\
(See user andrewmw94 at \url{https://github.com/mlpack/mlpack})
\end{itemize}
}

\vspace{5pt}

\cventry{Summer 2013}{NSF REU}{Florida International University}{Dr. Xin Sun}{}{
\begin{itemize}
\item[] Add a field to a packet header in OpenFlow to make administration of Software Defined Networks more easy and efficient (See [O'Neil HotSDN 2014])
\end{itemize}
}


\section{Skills}

\begin{itemize}
    \item[] \textbf{Languages:} C, C++, Java, \LaTeX, Python, Rust, Lean, Dafny, Coq, Matlab, OCaml
    \vspace{1em}
    \item[] \textbf{Frameworks:} ROS, Linux, PRISM, LLVM, Z3, OMPL, Tensorflow, Keras, Docker, Git
\end{itemize}

\section{Awards}

\vspace{6pt}

\begin{itemize}

\item[]{2019 ICRA Best Paper in Cognitive Robotics}

\vspace{6pt}

\item[]{2017 NASA Space Technology Research Fellowship}

\vspace{6pt}

\item[]{2017 NSF Graduate Research Fellowship Program Honorable Mention}

\vspace{6pt}

\item[]{TAROS 2015 Best Student Paper Award.}

\vspace{6pt}

\item[]{CRA Outstanding Undergraduate Researcher.  Honorable Mention 2016.}

\vspace{6pt}

\item[]{Best Poster Presentation award Florida International University Computer Science REU 2013.}

\vspace{6pt}

\item[]{Winner of CUA Math Contest Fall 2012 - Spring 2016}

\end{itemize}

\section{Publications}

\vspace{6pt}

\begin{itemize}

\item[]{T. Pan, A. M. Wells, R. Shome and L. E. Kavraki, ``Multi-Robot Task and Motion Planning,'' IEEE Intl. Conf. on Intelligent Robots and Systems, 2021, under review}

\vspace{1em} 

\item[] {A. M. Wells, Z. Kingston, M. Lahijanian, L. E. Kavraki and M. Y. Vardi, ``Probabilsitc synthesis for Robotic Manipulation,'' IEEE Intl. Conf. on Roboticts and Automation, 2021, to appear}

\vspace{1em} 

\item[]{A. M. Wells, M. Lahijanian, L. E. Kavraki, and M. Y. Vardi, “LTLf Synthesis on Probabilistic Systems,” Electronic Proceedings in Theoretical Computer Science, vol. 326, pp. 166–181, Sep. 2020.}

\vspace{1em} 

\item[]{T. Pan, C. K. Verginis, A. M. Wells, D.V. Dimarogonas and L. E. Kavraki, “Augmenting Control Policies with Motion Planning for Robust and Safe Multi-robot Navigation,” in IEEE Intl. Conf. on Intelligent Robots and Systems, 2020. To appear.}

\vspace{1em} 

\item[]{Z. Kingston, A. M. Wells, M. Moll, and L. E. Kavraki, “Informing Multi-Modal Planning with Synergistic Discrete Leads,” in IEEE Intl. Conf. on Robotics and Automation, 2020, pp. 3199–3205.}

\vspace{1em} 

\item[]{He, Keliang, Wells, Andrew M., Kavraki, Lydia E. and Vardi, Moshe. Y.  “Efficient Symbolic Reactive Synthesis for Finite-Horizon Tasks,” in IEEE Intl. Conf. on Robotics and Automation, 2019. (\textbf{Best Paper in Cognitive Robotics})}

\vspace{1em} 

\item[]{Wells, Andrew M., Dantam, Neil T., Shrivastava, Anshumali and Kavraki, Lydia E.  “Learning Feasibility for Task and Motion Planning in Tabletop Environments,” IEEE Robotics and Automation Letters, 2019. IEEE Robotics and Automation Letters, vol. 4, no. 2, pp. 285–292, Apr. 2019.}

\vspace{1em} 

\item[]{Wells, Andrew and Plaku, Erion.  “Adaptive Sampling Based Motion Planning for Mobile Robots with Differential Constraints.” Springer LNCS Towards Autonomous Robotic Systems, vol. 9287, pp. 283–295 \url{http://link.springer.com/chapter/10.1007%2F978-3-319-22416-9_32} (\textbf{Best Student Paper Award})}

\vspace{1em} 

\item[]{O’Neil, Michael, Wells, Andrew and Sun, Xin. “Towards a novel and efficient packet identifier design for SDN” HotSDN ‘14 Proceedings of the third workshop on Hot topics in software defined networking, pp. 223-224. \url{http://dl.acm.org/citation.cfm?id=2620728.2620775}}

\end{itemize}

% % \section{Talks}

% \vspace{6pt}
 
% \begin{itemize}

% \item{“Reactive Synthesis for Robot Manipulation via Binary Decision Diagrams.” Ann and H.J. Smead Aerospace Engineering Sciences, College of Engineering \& Applied Science, University of Colorado Boulder 2019.}

% \vspace{6pt}

% \item{“Sampling Based Motion Planning with Kinodynamics” at Pracsys Laboratory, Computational Biomedicine Imaging and Modeling Center, Rutgers University, 2015.}

% \end{itemize}

% \section{Service}

I have reviewed articles for the following:
\vspace{6pt}
\begin{itemize}

\item[] International Conference on Robotics and Automation (2020, 2021)
\vspace{6pt}
\item[] Intelligent Robots and Systems (2020)
\vspace{6pt}
\item[] Robotics and Automation Letters (2020, 2019, 2018)
\vspace{6pt}
\item[] Frontiers (2020)

\end{itemize}

\vspace{1em}
I have also served on the Rice Computer Science Graduate Student Association as treasurer (1 year) and as public relations officer (2 years).

% \section{Teaching}

\vspace{6pt}
 
\begin{itemize}

\item[]{Teaching Assistant for 6 classes: ``Introduction to Algorithmic Robotics,'' ``Automata, Formal Languages and Computing,'' ``Reasoning about Algorithms'' and ``Statistics, Computing and Data Science''}

\vspace{6pt}

\item[]{Taught a 2 semester highschool Computer Science class at University of Saint Thomas}

\vspace{6pt}

\item[]{Coach department-wide graduate student seminar on communication for wider audiences}

\end{itemize}

% \section{Mentoring}

\vspace{6pt}
 
\begin{itemize}

\item[]{Mentor two undergraduates in research-for-credit class on robotics}

\vspace{6pt}

\item[]{Mentor Ph.D. student for two years}

\end{itemize}

% Publications from a BibTeX file without multibib
%  for numerical labels: \renewcommand{\bibliographyitemlabel}{\@biblabel{\arabic{enumiv}}}% CONSIDER MERGING WITH PREAMBLE PART
%  to redefine the heading string ("Publications"): \renewcommand{\refname}{Articles}
% \nocite{*}
% \bibliographystyle{plain}
% \bibliography{publications}                        % 'publications' is the name of a BibTeX file

% Publications from a BibTeX file using the multibib package
%\section{Publications}
%\nocitebook{book1,book2}
%\bibliographystylebook{plain}
%\bibliographybook{publications}                   % 'publications' is the name of a BibTeX file
%\nocitemisc{misc1,misc2,misc3}
%\bibliographystylemisc{plain}
%\bibliographymisc{publications}                   % 'publications' is the name of a BibTeX file

%-----       letter       ---------------------------------------------------------

\end{document}


%% end of file `template.tex'.

%% start of file `template.tex'.
%% Copyright 2006-2013 Xavier Danaux (xdanaux@gmail.com).
%
% This work may be distributed and/or modified under the
% conditions of the LaTeX Project Public License version 1.3c,
% available at http://www.latex-project.org/lppl/.


\documentclass[11pt,a4paper,sans]{moderncv}        % possible options include font size ('10pt', '11pt' and '12pt'), paper size ('a4paper', 'letterpaper', 'a5paper', 'legalpaper', 'executivepaper' and 'landscape') and font family ('sans' and 'roman')

% modern themes
\moderncvstyle{banking}                            % style options are 'casual' (default), 'classic', 'oldstyle' and 'banking'
\moderncvcolor{blue}                                % color options 'blue' (default), 'orange', 'green', 'red', 'purple', 'grey' and 'black'
%\renewcommand{\familydefault}{\sfdefault}         % to set the default font; use '\sfdefault' for the default sans serif font, '\rmdefault' for the default roman one, or any tex font name
%\nopagenumbers{}                                  % uncomment to suppress automatic page numbering for CVs longer than one page

% character encoding
\usepackage[utf8]{inputenc}                       % if you are not using xelatex ou lualatex, replace by the encoding you are using
%\usepackage{CJKutf8}                              % if you need to use CJK to typeset your resume in Chinese, Japanese or Korean

% adjust the page margins
\usepackage[scale=0.75]{geometry}
%\setlength{\hintscolumnwidth}{3cm}                % if you want to change the width of the column with the dates
%\setlength{\makecvtitlenamewidth}{10cm}           % for the 'classic' style, if you want to force the width allocated to your name and avoid line breaks. be careful though, the length is normally calculated to avoid any overlap with your personal info; use this at your own typographical risks...

\usepackage{import}

%\usepackage{hyperref}
%\urlstyle{same}

% personal data
\name{Andrew}{Wells}
\title{R\'{e}sum\'{e}}                               % optional, remove / comment the line if not wanted
%\address{my address, line 1, line 2, line 3, postcode}{}{}% optional, remove / comment the line if not wanted; the "postcode city" and and "country" arguments can be omitted or provided empty
\phone[mobile]{+845 702 3699}                   % optional, remove / comment the line if not wanted
%\phone[fixed]{01234 123456}                    % optional, remove / comment the line if not wanted
%\phone[fax]{+3~(456)~789~012}                      % optional, remove / comment the line if not wanted
\email{andrew.wells@rice.edu}                               % optional, remove / comment the line if not wanted
\homepage{www.andrewmwells.com}                         % optional, remove / comment the line if not wanted
%\extrainfo{additional information}                 % optional, remove / comment the line if not wanted
%\photo[64pt][0.4pt]{picture}                       % optional, remove / comment the line if not wanted; '64pt' is the height the picture must be resized to, 0.4pt is the thickness of the frame around it (put it to 0pt for no frame) and 'picture' is the name of the picture file
%\quote{Some quote}                                 % optional, remove / comment the line if not wanted

% to show numerical labels in the bibliography (default is to show no labels); only useful if you make citations in your resume
%\makeatletter
%\renewcommand*{\bibliographyitemlabel}{\@biblabel{\arabic{enumiv}}}
%\makeatother
%\renewcommand*{\bibliographyitemlabel}{[\arabic{enumiv}]}% CONSIDER REPLACING THE ABOVE BY THIS

% bibliography with mutiple entries
%\usepackage{multibib}
%\newcites{book,misc}{{Books},{Others}}
%----------------------------------------------------------------------------------
%            content
%----------------------------------------------------------------------------------
\begin{document}
%\begin{CJK*}{UTF8}{gbsn}                          % to typeset your resume in Chinese using CJK
%-----       resume       ---------------------------------------------------------
\makecvtitle

\small{I am a $5^{th}$ year Ph.D. Student at Rice University advised by Dr. Lydia Kavraki. I will graduate this year. In 2017, I recieved a NASA Space Technology Research Fellowship (now called Space Technology Research Grants (NSTRG)) to conduct research in Task and Motion Planning (TMP) and in Formal Methods (FM) for Robotics. This research includes numerous publications, as well as implementations on physical robots. The research spans an array of topics including (a) applications of Machine Learing to TMP to improve scalability, (b) theoretical work in FM to ensuring or verifying correctness and safety, (c) applications of FM to robotics to enhance robot autonomy while ensuring safety and (d) research in multi-robot planning and control as well as multi-robot task and motion planning. As a NSTRG fellow, my research is relevant to NASA Technology Roadmap 4 (specifically, 4.2.6.1-4.2.6.2, 4.3.2, 4.3.4, 4.4.3-4.4.4, 4.4.7, 4.5.2-4.5.8 and 4.7.2-4.7.5). I have helped individuals at NASA apply my research, collaborating with the Robotics and Intelligence for Human Spacecraft (RIHS) team at the Lyndon B. Johnson Space Center (JSC) and with researchers at NASA Ames. On the hardware side, I have experience on Universal Robotics UR5 robots as well as on NASA's Robonaut 2. I have collaborated with the RIHS team at JSC since 2017 as well as researchers at other labs and universities throughout the world. I have experience leading two small projects of three or four researchers while mentoring a junior Ph.D. student at Rice University, including choosing high-level goals, assigning tasks, and scheduling subgoals to ensure our team meets deadlines.}

\section{Research Experience}

\vspace{6pt}

\begin{itemize}

\item{\cventry{August 2016--Present}{Research Assistant}{Rice University}{Dr. Kavraki}{}{\vspace{3pt}As a Ph.D. student at Rice, I conduct research in robotics, specifically in task and motion planning and in formal methods for robotics. I have conducted research on a variety of topics including: Applications of Machine Learing to TMP to improve scalability; theoretical work in Formal Methods to ensuring or verifying correctness and safety; applied work in using FM to enhance robot autonomy while ensuring safety; and research in multi-robot planning and control as well as multi-robot task and motion planning. Current work examines: improving scalability, particularly with multiple robots; using stochastic games to model the robot collaboration more accurately; combining machine learning with formal methods to provide formal guarantees on the result.}}

\vspace{6pt}

\item{\cventry{May 2016--August 2016}{Google Summer of Code}{LLVM/Linux}{Jan-Simon Moller}{}{\vspace{3pt}Worked on the Clang static-analyzer for the Linux Kernel. The project was hosted under the linux foundation, the specific project being LLVMLinux. I implemented checkers for the linux kernel.  (Exact contributions can be found for user andrewmw94 at \url{https://github.com/andrewmw94/llvm_clang_GSoC})}}

\vspace{6pt}

\item{\cventry{September 2013--May 2016}{Research Assistant}{Catholic University of America}{Dr. Plaku}{}{\vspace{3pt}Motion Planning: Conducted research in robot motion planning with an emphasis on motion planning for high-dimensional systems with nonlinear dynamics.  Discrete algorithms are combined with sampling-based motion planning algorithms to increase the performance.  The work shows that it is possible to gain significant speedups over other state of the art methods (eg. RRT and SYCLOP) by incorporating feedback from the continuous space into the discrete space.  This work is resulted in a publication.
Linear Temporal Logic Multi-Robot Planning:  Conducted research in motion planning involving multiple robots, multiple goals, and requirements for reaching the goals specified by a proposition of Linear Temporal Logic.  The continuous planner is guided by a discrete planner that must both satisfy the proposition and find short trajectories for the robots while avoiding collisions between robots.}}

\vspace{6pt}

\item{\cventry{May 2015--August 2015}{NSF REU}{DIMACS, Rutgers University}{Dr. Bekris}{}{\vspace{3pt}Probabilistic Near-Optimality:  Bekris and Dobson had proved finite time Probabilistic Near-Optimality of a planning algorithm similar to PRM; however, their proof did not include tree-based planners such as RRT.  I extended the proof to include such planners.  This work will be published as an addendum to their paper.
Motion Planning using Homotopic Constraints:  Conducted research in robotics motion planning for dynamic systems.  This further explores the advantages to be gained by coupling discrete algorithms with continuous planners.  The focus in this work is on finding shortest paths in different homotopic classes and using these to guide a sampling-based planner, by choosing a homotopic class and following the corresponding shortest path.}}

\vspace{6pt}

\item{\cventry{May 2014--August 2014}{Google Summer of Code}{MLPACK}{Dr. Curtin}{}{\vspace{3pt}Work on the MLPACK library for Machine Learning, specifically on structures and algorithms for nearest-neighbor searches.  My contributions were implementing R-trees and variants as well as associated search algorithms.  (Exact contributions can be found for user andrewmw94 at \url{https://github.com/mlpack/mlpack})}}

\vspace{6pt}

\item{\cventry{May 2013--August 2013}{NSF REU}{Florida International University}{Dr. Sun}{}{\vspace{3pt}Research in Software Defined Networking.  The work centered around the difficulties facing administrators of large networks and the possibility of adding a field to a packet header so that administration could be done easily and rules enforced efficiently. This work resulted in a publication.}}

\end{itemize}

\section{Education}

\vspace{3pt}

\begin{itemize}

\item{\cventry{August 2016--Present}{Ph.D. student advised by Dr. Lydia Kavraki}{Rice University}{}{}{}}

\item{\cventry{2012--2016}{BS in Computer Science}{Catholic University of America}{}{\textit{Magna Cum Laude}}{}}  % arguments 3 to 6 can be left empty
\item{\cventry{2012--2016}{Ph.B. in Philosophy}{Catholic University of America}{}{\textit{Magna Cum Laude}}{}}

\end{itemize}


\section{Awards}

\vspace{6pt}

\begin{itemize}

\item{2019 ICRA Best Paper in Cognitive Robotics}

\vspace{6pt}

\item{2017 NASA Space Technology Research Fellowship}

\vspace{6pt}

\item{2017 NSF Graduate Research Fellowship Program Honorable Mention}

\vspace{6pt}

\item{TAROS 2015 Best Student Paper Award.}

\vspace{6pt}

\item{CRA Outstanding Undergraduate Researcher.  Honorable Mention 2016.}

\vspace{6pt}

\item{Best Poster Presentation award Florida International University Computer Science REU program 2013.}

\vspace{6pt}

\item{Winner of CUA Math Contest Fall 2012 - Spring 2016}

\end{itemize}

\section{Publications}

\vspace{6pt}

\begin{itemize}

\item{A. M. Wells, M. Lahijanian, L. E. Kavraki, and M. Y. Vardi, “LTLf Synthesis on Probabilistic Systems,” Electronic Proceedings in Theoretical Computer Science, vol. 326, pp. 166–181, Sep. 2020.}

\item{T. Pan, C. K. Verginis, A. M. Wells, D.V. Dimarogonas and L. E. Kavraki, “Augmenting Control Policies with Motion Planning for Robust and Safe Multi-robot Navigation,” in IEEE Intl. Conf. on Intelligent Robots and Systems, 2020. To appear.}

\item{Z. Kingston, A. M. Wells, M. Moll, and L. E. Kavraki, “Informing Multi-Modal Planning with Synergistic Discrete Leads,” in IEEE Intl. Conf. on Robotics and Automation, 2020, pp. 3199–3205.}

\item{Wells, Andrew M., Dantam, Neil T., Shrivastava, Anshumali and Kavraki, Lydia E.  “Learning Feasibility for Task and Motion Planning in Tabletop Environments,” IEEE Robotics and Automation Letters, 2019. IEEE Robotics and Automation Letters, vol. 4, no. 2, pp. 285–292, Apr. 2019.}

\item{He, Keliang, Wells, Andrew M., Kavraki, Lydia E. and Vardi, Moshe. Y.  “Efficient Symbolic Reactive Synthesis for Finite-Horizon Tasks,” in IEEE Intl. Conf. on Robotics and Automation, 2019. (Best Paper in Cognitive Robotics)}

\item{Wells, Andrew and Plaku, Erion.  “Adaptive Sampling Based Motion Planning for Mobile Robots with Differential Constraints.” Springer LNCS Towards Autonomous Robotic Systems, vol. 9287, pp. 283–295 \url{http://link.springer.com/chapter/10.1007%2F978-3-319-22416-9_32} (Best Student Paper Award)}

\item{O’Neil, Michael, Wells, Andrew and Sun, Xin. “Towards a novel and efficient packet identifier design for SDN” HotSDN ‘14 Proceedings of the third workshop on Hot topics in software defined networking, pp. 223-224. \url{http://dl.acm.org/citation.cfm?id=2620728.2620775}}

\end{itemize}

\section{Talks}

\vspace{6pt}
 
\begin{itemize}

\item{“Reactive Synthesis for Robot Manipulation via Binary Decision Diagrams.” Ann and H.J. Smead Aerospace Engineering Sciences, College of Engineering \& Applied Science, University of Colorado Boulder 2019.}

\vspace{6pt}

\item{“Sampling Based Motion Planning with Kinodynamics” at Pracsys Laboratory, Computational Biomedicine Imaging and Modeling Center, Rutgers University, 2015.}

\end{itemize}

% Publications from a BibTeX file without multibib
%  for numerical labels: \renewcommand{\bibliographyitemlabel}{\@biblabel{\arabic{enumiv}}}% CONSIDER MERGING WITH PREAMBLE PART
%  to redefine the heading string ("Publications"): \renewcommand{\refname}{Articles}
\nocite{*}
\bibliographystyle{plain}
\bibliography{publications}                        % 'publications' is the name of a BibTeX file

% Publications from a BibTeX file using the multibib package
%\section{Publications}
%\nocitebook{book1,book2}
%\bibliographystylebook{plain}
%\bibliographybook{publications}                   % 'publications' is the name of a BibTeX file
%\nocitemisc{misc1,misc2,misc3}
%\bibliographystylemisc{plain}
%\bibliographymisc{publications}                   % 'publications' is the name of a BibTeX file

%-----       letter       ---------------------------------------------------------

\end{document}


%% end of file `template.tex'.
